\documentclass[11pt]{article}

\newcommand{\yourname}{Kevin Zhang}

\def\comments{0}

%format and packages

%\usepackage{algorithm, algorithmic}
\usepackage{algpseudocode}
\usepackage{amsmath, amssymb, amsthm}
\usepackage{enumerate}
\usepackage{enumitem}
\usepackage{framed}
\usepackage{verbatim}
\usepackage[margin=1.0in]{geometry}
\usepackage{microtype}
\usepackage{kpfonts}
\usepackage{palatino}
	\DeclareMathAlphabet{\mathtt}{OT1}{cmtt}{m}{n}
	\SetMathAlphabet{\mathtt}{bold}{OT1}{cmtt}{bx}{n}
	\DeclareMathAlphabet{\mathsf}{OT1}{cmss}{m}{n}
	\SetMathAlphabet{\mathsf}{bold}{OT1}{cmss}{bx}{n}
	\renewcommand*\ttdefault{cmtt}
	\renewcommand*\sfdefault{cmss}
	\renewcommand{\baselinestretch}{1.06}
\usepackage[usenames,dvipsnames]{xcolor}
\definecolor{DarkGreen}{rgb}{0.15,0.5,0.15}
\definecolor{DarkRed}{rgb}{0.6,0.2,0.2}
\definecolor{DarkBlue}{rgb}{0.2,0.2,0.6}
\definecolor{DarkPurple}{rgb}{0.4,0.2,0.4}
\usepackage[pdftex]{hyperref}
\hypersetup{
	linktocpage=true,
	colorlinks=true,				% false: boxed links; true: colored links
	linkcolor=DarkBlue,		% color of internal links
	citecolor=DarkBlue,	% color of links to bibliography
	urlcolor=DarkBlue,		% color of external links
}

\usepackage[boxruled,vlined,nofillcomment]{algorithm2e}
	\SetKwProg{Fn}{Function}{\string:}{}
	\SetKwFor{While}{While}{}{}
	\SetKwFor{For}{For}{}{}
	\SetKwIF{If}{ElseIf}{Else}{If}{:}{ElseIf}{Else}{:}
	\SetKw{Return}{Return}
	

%enclosure macros
\newcommand{\paren}[1]{\ensuremath{\left( {#1} \right)}}
\newcommand{\bracket}[1]{\ensuremath{\left\{ {#1} \right\}}}
\renewcommand{\sb}[1]{\ensuremath{\left[ {#1} \right\]}}
\newcommand{\ab}[1]{\ensuremath{\left\langle {#1} \right\rangle}}

%probability macros
\newcommand{\ex}[2]{{\ifx&#1& \mathbb{E} \else \underset{#1}{\mathbb{E}} \fi \left[#2\right]}}
\newcommand{\pr}[2]{{\ifx&#1& \mathbb{P} \else \underset{#1}{\mathbb{P}} \fi \left[#2\right]}}
\newcommand{\var}[2]{{\ifx&#1& \mathrm{Var} \else \underset{#1}{\mathrm{Var}} \fi \left[#2\right]}}

%useful CS macros
\newcommand{\poly}{\mathrm{poly}}
\newcommand{\polylog}{\mathrm{polylog}}
\newcommand{\zo}{\{0,1\}}
\newcommand{\pmo}{\{\pm1\}}
\newcommand{\getsr}{\gets_{\mbox{\tiny R}}}
\newcommand{\card}[1]{\left| #1 \right|}
\newcommand{\set}[1]{\left\{#1\right\}}
\newcommand{\negl}{\mathrm{negl}}
\newcommand{\eps}{\varepsilon}
\DeclareMathOperator*{\argmin}{arg\,min}
\DeclareMathOperator*{\argmax}{arg\,max}
\newcommand{\eqand}{\qquad \textrm{and} \qquad}
\newcommand{\ind}[1]{\mathbb{I}\{#1\}}
\newcommand{\sslash}{\ensuremath{\mathbin{/\mkern-3mu/}}}

%mathbb
\newcommand{\N}{\mathbb{N}}
\newcommand{\R}{\mathbb{R}}
\newcommand{\Z}{\mathbb{Z}}
%mathcal
\newcommand{\cA}{\mathcal{A}}
\newcommand{\cB}{\mathcal{B}}
\newcommand{\cC}{\mathcal{C}}
\newcommand{\cD}{\mathcal{D}}
\newcommand{\cE}{\mathcal{E}}
\newcommand{\cF}{\mathcal{F}}
\newcommand{\cL}{\mathcal{L}}
\newcommand{\cM}{\mathcal{M}}
\newcommand{\cO}{\mathcal{O}}
\newcommand{\cP}{\mathcal{P}}
\newcommand{\cQ}{\mathcal{Q}}
\newcommand{\cR}{\mathcal{R}}
\newcommand{\cS}{\mathcal{S}}
\newcommand{\cU}{\mathcal{U}}
\newcommand{\cV}{\mathcal{V}}
\newcommand{\cW}{\mathcal{W}}
\newcommand{\cX}{\mathcal{X}}
\newcommand{\cY}{\mathcal{Y}}
\newcommand{\cZ}{\mathcal{Z}}

%theorem macros
\newtheorem{thm}{Theorem}
\newtheorem{lem}[thm]{Lemma}
\newtheorem{fact}[thm]{Fact}
\newtheorem{clm}[thm]{Claim}
\newtheorem{rem}[thm]{Remark}
\newtheorem{coro}[thm]{Corollary}
\newtheorem{prop}[thm]{Proposition}
\newtheorem{conj}[thm]{Conjecture}

\theoremstyle{definition}
\newtheorem{defn}[thm]{Definition}
\newtheoremstyle{case}{}{}{}{}{}{:}{ }{}
\theoremstyle{case}
\newtheorem{case}{Case}

\newcommand{\instructor}{Jonathan Ullman}
\newcommand{\hwnum}{1}
\newcommand{\hwdue}{Tuesday September 18 at 11:59pm via \href{https://gradescope.com/courses/13812}{Gradescope}}

\theoremstyle{theorem}
\newtheorem{prob}{Problem}
\newtheorem{sol}{Solution}

\definecolor{cit}{rgb}{0.05,0.2,0.45} 
\newcommand{\solution}{\medskip\noindent{\color{DarkBlue}\textbf{Solution:}}}

\begin{document}
{\large
\noindent Name: \yourname}

\vspace{15pt}

\begin{prob}\end{prob}

Tseitin's Rules: 
\indent
\begin{enumerate}
\item $ac = ca$
\item $ad = da$
\item $bc = cb$
\item $bd = db$
\item $ce = eca$
\item $de = edb$
\item $cca = ccae$
\end{enumerate}

\begin{enumerate}[label=(\alph*)]
\item
$bccdbc \neq cbabd$ \\
The pair is not equivalent, because $bccdbc$ is a 6-letter word, and $cbabd$ is a 5-letter word, so we would need to remove a letter to go from left to right. 
Rules 1-4 are swap rules, and cannot remove a letter. Rules 5 and 6 require an $e$, 
which the left side does not have, nor can we use Rule 7 to create one (because there is no $a$ on the left side). \\
Therefore, the two are not equivalent. 

\item
$cadbcedb \neq caccaedb$ \\
The pair is not equivalent, because $cadbcedb$ has two $d$'s, and $caccaedb$ has only one. None of the rules can remove or create a $d$:\\
\indent Rules 1-4 are swap rules, and does not create or remove letters\\
\indent Rule 5 creates an $a$ when going from left to right.\\
\indent Rule 6 creates an $b$ when going from left to right.\\
\indent Rule 7 creates an $e$ when going from left to right.\\
Therefore, the two are not equivalent.

\item
\begin{align*}
  aecdab &\stackrel{?}{=} cade \\
         &\stackrel{?}{=} caedb \space \text{(Rule 6: $de = edb$)} \\
         &\stackrel{?}{=} acedb \space \text{(Rule 1: $ca = ac$)} \\
         &\stackrel{?}{=} aecadb \space \text{(Rule 5: $ce = eca$)} \\
         &\stackrel{?}{=} aecdab \space \text{(Rule 2: $ad = da$)} \\
  aecdab &= aecdab \checkmark
\end{align*}\\
Therefore, the two are equivalent.
\end{enumerate}

\newpage

\begin{prob}\end{prob}
\begin{enumerate}[label=(\alph*)]
\item
If February has 30 days, then 7 is an odd number. \\
February has 30 days $=$ False \\
7 is an odd number $=$ True \\
False $\rightarrow$ True $=$ \textbf{True}

\item
If January has 31 days, then 7 is an even number. \\
January has 31 days $=$ True \\
7 is an even number $=$ False \\
True $\rightarrow$ False $=$ \textbf{False}

\item
If 7 is an odd number, then February does not have 30 days \\
7 is an odd number $=$ True \\
February does not have 30 days $=$ True \\
True $\rightarrow$ True $=$ \textbf{True}

\item
If 7 is an even number, then January has exactly 28 days \\
7 is an even number $=$ False \\
January has exactly 28 days $=$ False \\
False $\rightarrow$ False $=$ \textbf{True}
\end{enumerate}

\begin{prob}\end{prob}
\begin{proof}
Suppose $\sqrt{x}$ is rational. Then $\sqrt{x} \cdot \sqrt{x}$ is also rational. $x = \sqrt{x}^{2}$, so $x$ must also be rational. \\
Therefore, by the contrapositive, if $x$ is irrational, then $\sqrt{x}$ must be irrational.

\end{proof}

\newpage

\begin{prob}\end{prob}
\begin{enumerate}[label=(\alph*)]
\item
$\forall x \hspace{5px} (x \cdot 2 \neq x \cdot 3)$ is False when $x = 0$ \\
\begin{align*}
  &\neg ( \forall x \hspace{5px} (x \cdot 2 \neq x \cdot 3) ) \\
  &\exists x \hspace{5px} \neg (x \cdot 2 \neq x \cdot 3) \\
  &\exists x \hspace{5px} (x \cdot 2 = x \cdot 3)
\end{align*}

\item
$\exists x \hspace{5px} (x + 2 = x + 3)$ is False. \\
\begin{align*}
  &\neg (\exists x \hspace{5px} (x + 2 = x + 3)) \\
  &\forall x \hspace{5px} \neg (x + 2 = x + 3) \\
  &\forall x \hspace{5px} (x + 2 \neq x + 3)
\end{align*}

\item
$\forall x \hspace{5px} (x^2 \neq x)$ is False when $x = 0$ or $x = 1$\\
\begin{align*}
  &\neg (\forall x \hspace{5px} (x^2 \neq x)) \\
  &\exists x \hspace{5px} \neg (x^2 \neq x) \\
  &\exists x \hspace{5px} (x^2 = x)
\end{align*}

\item
$\exists x \hspace{5px} (5 \leq x < 6)$ is True. $x = 5$ is possible. \\
\begin{align*}
  &\neg (\exists x \hspace{5px} (5 \leq x < 6)) \\
  &\forall x \hspace{5px} \neg (5 \leq x < 6) \\
  &\forall x \hspace{5px} \neg (5 \leq x \land x < 6) \\
  &\forall x \hspace{5px} \neg (5 \leq x) \lor \neg (x < 6) \\
  &\forall x \hspace{5px} (5 > x) \lor (x \geq 6)
\end{align*}

\end{enumerate}
\newpage

\begin{prob}\end{prob}
\begin{enumerate}[label=(\alph*)]
\item
$\exists x \hspace{5px} CS(x) \land \neg Charlie(x)$
\item
$\forall x \hspace{5px} Charlie(x) \rightarrow CS(x)$
\item
$\exists x_1 \exists x_2 \hspace{5px} x_1 \neq x_2 \land Charlie(x_1) \land Charlie(x_2)$
\end{enumerate}

\begin{prob}\end{prob}

\begin{enumerate}[label=(\alph*)]
\item 
$\exists x \forall y \hspace{5px} Knows(x, y)$

\item
$\exists x \forall y \hspace{5px} CS(y) \land Knows(x, y)$

\item
$\forall x \exists y \hspace{5px} CS(x) \land \neg CS(y) \land Knows(x, y)$

\item
$\forall x \exists y \forall z \hspace{5px} Knows(x, y) \land CS(z) \land \neg Knows(y, z)$
\end{enumerate}

\begin{prob}\end{prob}
\begin{enumerate}[label=(\alph*)]
\item
$\{1, 2, 3, 4\}$

\item
$\{2\}$

\item
$\{(2, 1), (2, 2), (2, 4), (3, 1), (3, 2), (3, 4)\}$

\item
$\{3\}$

\item
$\varnothing$
\end{enumerate}

\begin{prob}\end{prob}
\begin{enumerate}[label=(\alph*)]
\item
As $i$ goes to infinity, the interval shrinks to $[-0, 0]$, so the intersection would be $\{0\}$.

\item
As $i$ goes to infinity, the interval is always shrinking to $[-0, 0]$, so the union would be largest interval: $[-1, 1]$ or $S_1$.

\end{enumerate}

\begin{prob}\end{prob}
\begin{enumerate}[label=(\alph*)]
\item
$\{(1, x, u), (1, x, v), (1, y, u), (1, y, v), (2, x, u), (2, x, v), (2, y, u), (2, y, v), (3, x, u), (3, x, v), (3, y, u), (3, y, v)\}$

\item
$|A \times B \times C \times D \times E| = |A| \times |B| \times |C| \times |D| \times |E| = 60$
\end{enumerate}

\newpage

\begin{prob}\end{prob}
\begin{enumerate}
\item
$(\varnothing, \varnothing)$
\item
$(\varnothing, \{ 3 \})$
\item
$(\varnothing, \{ 4 \})$
\item
$(\varnothing, \{ 3, 4 \})$
\item
$(\{ 3 \}, \{ 3 \})$
\item
$(\{ 3 \}, \{ 3, 4 \})$
\end{enumerate}

\begin{prob}\end{prob}
(Diagrams from Left to Right)
\begin{enumerate}
\item
Is a function; every $x$ produces a unique $y$.
\item
Is not a function; there are some $x$ with no result $y$.
\item
Is a function; every $x$ produces a unique $y$.
\item
Is not a function; there are some $x$ that produce multiple $y$.
\end{enumerate}

\begin{prob}\end{prob}
(Diagrams from Left to Right)
\begin{enumerate}
\item
injection
\item
none
\item
bijection
\item
surjection
\end{enumerate}

\end{document}
