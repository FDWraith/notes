\documentclass[11pt]{article}

\newcommand{\yourname}{Kevin Zhang}

\def\comments{0}

%format and packages

%\usepackage{algorithm, algorithmic}
\usepackage{forest}
\usepackage{tikz}
\usepackage{algpseudocode}
\usepackage{amsmath, amssymb, amsthm}
\usepackage{tcolorbox}
\usepackage{enumerate}
\usepackage{enumitem}
\usepackage{framed}
\usepackage{verbatim}
\usepackage[margin=1.0in]{geometry}
\usepackage{microtype}
\usepackage{kpfonts}
\usepackage{palatino}
	\DeclareMathAlphabet{\mathtt}{OT1}{cmtt}{m}{n}
	\SetMathAlphabet{\mathtt}{bold}{OT1}{cmtt}{bx}{n}
	\DeclareMathAlphabet{\mathsf}{OT1}{cmss}{m}{n}
	\SetMathAlphabet{\mathsf}{bold}{OT1}{cmss}{bx}{n}
	\renewcommand*\ttdefault{cmtt}
	\renewcommand*\sfdefault{cmss}
	\renewcommand{\baselinestretch}{1.06}

\usepackage[boxruled,vlined,nofillcomment]{algorithm2e}
	\SetKwProg{Fn}{Function}{\string:}{}
	\SetKwFor{While}{While}{}{}
	\SetKwFor{For}{For}{}{}
	\SetKwIF{If}{ElseIf}{Else}{If}{:}{ElseIf}{Else}{:}
	\SetKw{Return}{Return}
	

%enclosure macros
\newcommand{\paren}[1]{\ensuremath{\left( {#1} \right)}}
\newcommand{\bracket}[1]{\ensuremath{\left\{ {#1} \right\}}}
\renewcommand{\sb}[1]{\ensuremath{\left[ {#1} \right\]}}
\newcommand{\ab}[1]{\ensuremath{\left\langle {#1} \right\rangle}}

%probability macros
\newcommand{\ex}[2]{{\ifx&#1& \mathbb{E} \else \underset{#1}{\mathbb{E}} \fi \left[#2\right]}}
\newcommand{\pr}[2]{{\ifx&#1& \mathbb{P} \else \underset{#1}{\mathbb{P}} \fi \left[#2\right]}}
\newcommand{\var}[2]{{\ifx&#1& \mathrm{Var} \else \underset{#1}{\mathrm{Var}} \fi \left[#2\right]}}

%useful CS macros
\newcommand{\poly}{\mathrm{poly}}
\newcommand{\polylog}{\mathrm{polylog}}
\newcommand{\zo}{\{0,1\}}
\newcommand{\pmo}{\{\pm1\}}
\newcommand{\getsr}{\gets_{\mbox{\tiny R}}}
\newcommand{\card}[1]{\left| #1 \right|}
\newcommand{\set}[1]{\left\{#1\right\}}
\newcommand{\negl}{\mathrm{negl}}
\newcommand{\eps}{\varepsilon}
\DeclareMathOperator*{\argmin}{arg\,min}
\DeclareMathOperator*{\argmax}{arg\,max}
\newcommand{\eqand}{\qquad \textrm{and} \qquad}
\newcommand{\ind}[1]{\mathbb{I}\{#1\}}
\newcommand{\sslash}{\ensuremath{\mathbin{/\mkern-3mu/}}}
\newcommand{\pipe}{\hspace{3pt}|\hspace{3pt}}

%mathbb
\newcommand{\N}{\mathbb{N}}
\newcommand{\R}{\mathbb{R}}
\newcommand{\Z}{\mathbb{Z}}
%mathcal
\newcommand{\cA}{\mathcal{A}}
\newcommand{\cB}{\mathcal{B}}
\newcommand{\cC}{\mathcal{C}}
\newcommand{\cD}{\mathcal{D}}
\newcommand{\cE}{\mathcal{E}}
\newcommand{\cF}{\mathcal{F}}
\newcommand{\cL}{\mathcal{L}}
\newcommand{\cM}{\mathcal{M}}
\newcommand{\cO}{\mathcal{O}}
\newcommand{\cP}{\mathcal{P}}
\newcommand{\cQ}{\mathcal{Q}}
\newcommand{\cR}{\mathcal{R}}
\newcommand{\cS}{\mathcal{S}}
\newcommand{\cU}{\mathcal{U}}
\newcommand{\cV}{\mathcal{V}}
\newcommand{\cW}{\mathcal{W}}
\newcommand{\cX}{\mathcal{X}}
\newcommand{\cY}{\mathcal{Y}}
\newcommand{\cZ}{\mathcal{Z}}

%theorem macros
\newtheorem{thm}{Theorem}
\newtheorem{lem}[thm]{Lemma}
\newtheorem{fact}[thm]{Fact}
\newtheorem{clm}[thm]{Claim}
\newtheorem{rem}[thm]{Remark}
\newtheorem{coro}[thm]{Corollary}
\newtheorem{prop}[thm]{Proposition}
\newtheorem{conj}[thm]{Conjecture}

\theoremstyle{definition}
\newtheorem{defn}[thm]{Definition}
\newtheoremstyle{case}{}{}{}{}{}{:}{ }{}
\theoremstyle{case}
\newtheorem{case}{Case}

\theoremstyle{theorem}
\newtheorem{prob}{Problem}
\newtheorem{sol}{Solution}

\tikzset{every picture/.style={line width=0.75pt}} %set default line width to 0.75pt        

\begin{document}
{\large
\noindent Name: \yourname}

\vspace{15pt}

\begin{prob}\end{prob}

\begin{enumerate}[label=(\alph*)]

\item

$A \leq_T B$ is true. $A$ is already decidable, so we can ignore the oracle for $B$, no matter what $B$ is.

\item

$A \leq_m B$ is false. If $B = \varnothing$, there is no $f: \Sigma^* \rightarrow \Sigma^*$ that
can turn an element of $A$ into nothing.

\end{enumerate}

\begin{prob}\end{prob}

\begin{enumerate}[label=(\alph*)]

\item

$L_1$ is mapping-reducible to $A_{TM}$. We can define a $f$ as follows: 

\[ f(x) = \begin{cases}

    x & x \text{ is not an encoding} \langle M \rangle \\

    \langle M, s \rangle & \text{ if x is an encoding } \langle M \rangle \text{ and s is a string consisting only of 1's that is accepted by M} \\

    \end{cases}
\]

\item

$L_2$ is not mapping-reducible to $A_{TM}$. $A_{TM} = \{ \langle M, s  \rangle \pipe \text{ M is a TM that accepts s } \}$. 
Since the machines in $L_2$ does not accept anything, there is no way to turn all those machines into acceptors.

\end{enumerate}

\begin{prob}\end{prob}

Want to show $A \subseteq \Sigma^* \text{ is decidable} \leftrightarrow A \leq_m 0^*1^* $. 

\begin{enumerate}[label=(\alph*)]

\item

Prove $A \subseteq \Sigma^* \text { is decidable} \rightarrow A \leq_m 0^*1^*$.

Assume $A$ is decidable. Then, we can definte a $f$ as follows: 

\[ f(x) = \begin{cases} 
    
    1 & \text{ if x is accepted by A } \\

    0 & \text{ if x is rejected by A }    

    \end{cases}
\]

Thus, $A \leq_m 0^*1^*$.

\item

Prove $A \subseteq \Sigma^* \text { is decidable} \leftarrow A \leq_m 0^*1^*$.

Assume $A \leq_m 0^*1^*$. 

$0^*1^*$ is a regular expression, which means it can be represented with some DFA $D$. 

$D$ is decidable, because all DFAs are decidable.

$A \leq_m B \text{ and B is decidable } \rightarrow \text{ A is decidable }$ (Shown in class, Lec22ab, Slide 150).

Therefore, $A$ must be decidable.

\end{enumerate}

\newpage

\begin{prob}\end{prob}

Prove that $HALT$ is enumerably complete. From the definition of enumerably
complete, have two show two things:

\begin{enumerate}[label=(\arabic*)]

\item

Prove $HALT$ is enumerable / recognizable. We can do so by constructing a recognizer:

On given input $x$, if $x$ is not encoding $\langle M, w \rangle$, reject. If it is,
then run $M$ on input $w$. If $M$ accepts or rejects (halts), then $HALT$ will accept. 

\item

Prove for all $L \subseteq Sigma^*$, $L \leq_m HALT$.

We know that $A_{TM}$ is enumerably complete (hw10, pb 7). This means that for all $L \subseteq Sigma^*$, $L \leq_m A_{TM}$.

$A_{TM} \leq_m HALT$ was shown previously (Lec 22ab, Slide 148).

For any $L \subseteq \Sigma^*$, if $L \leq_m A_{TM}$ and $A_{TM} \leq_m HALT$, then $L \leq_m HALT$ (from class). 

Thus, for all $L \subseteq \Sigma^*$, $L \leq_m HALT$.

\end{enumerate}

\begin{prob}\end{prob}

$L = \{ w \pipe \text{either w = 0x for some } x \in A_{TM} \text{ or w = 1y for some } y \in \overline{A_{TM}}\}$.

\begin{enumerate}[label=(\arabic*)]

\item

Prove $L$ is not recognizable.

$\overline{A_{TM}} \leq_m L$. We can create a mapping function $f$ that adds a $1$ before every element of $\overline{A_{TM}}$.

$\overline{A_{TM}}$ is not recognizable. We can prove this because $A_{TM}$ is recognizable (shown in class), 
$A_{TM}$ is not decidable (shown in class), and $\overline{A_{TM}}$ cannot be recognizable because 
$A \text{ recognizable and } \bar{A} \text{ recognizable } \rightarrow A \text{ decidable }$ (shown in class). 

$A \leq_m B \text{ and A not recognizable } \rightarrow B \text{ is not recognizable }$ (Lec22ab, slide 150).

Thus, $L$ is not recognizable.

\item

Prove $\bar{L}$ is not recognizable.

We can use a similar proof. $\overline{A_{TM}} \leq_m \bar{L}$. We can create a mapping function $f$ that adds a $0$ before every element of $\overline{A_{TM}}$.

$\overline{A_{TM}}$ is not recognizable (See above).

$A \leq_m B \text{ and A not recognizable } \rightarrow B \text{ is not recognizable }$ (Lec22ab, slide 150).

Thus, $\bar{L}$ is not recognizable.

\end{enumerate}

\newpage

\begin{prob}\end{prob}

True. $0^*1^*$ is a regular language, which is in class $P$ (Lec 22c, Slide 7). Then, we
apply property 2 on Lec23, Slide 28 to show $L \in P$.

\begin{prob}\end{prob}

$K \stackrel{?}{=} \text{ recognizable }$. From Lecture 18c, we know that 
$A$ is recognizable iff there exists decidable language B such that for all $w \in \Sigma^*$:
$w \in A \leftrightarrow \exists s ( \langle w, s \rangle \in B)$. In this case, $B \in P$, which
makes B a decidable language. Therefore, $K$ is most similar to recognizable languages. Most likely,
$K \subseteq $ recognizable. 

$K$ cannot be decidable, because there is no constraint on $|s|$, so searching for $s$ could
potentially take forever, so $\text{decidable} \subseteq K$.

Thus, the full-chain might look like so:

$\text{ regular } \subseteq \text{ context-free } \subseteq \text{ P } \subseteq \text{ NP } \subseteq \text{ decidable } \subseteq \text{ K } \subseteq \text{ recognizable }$.  

\begin{prob}\end{prob}

Show that $ISO \in NP$. 

\begin{enumerate}[label=(\arabic*)]

\item

$ISO$ can be solved by non-deterministically assigning numbers to nodes of each graph, 
sorting the adjacency matrixes of the graphs (by node-number), and then comparing the two results. 
Differently-sized adjacency matrixes will be rejected outright.

\item

$ISO$ can be verified in polynomial time. Compare the resulting adjacency matricies of $G$ and $H$,
to see if both graphs have the same verticies + edges. This will take $O(n^2)$. 

\end{enumerate}

\end{document}
